\chapter{Objetivos}
%Dentro dos objetivos compreendidos como gerais, busca-se apresentar os resultados obtidos durante o período de duração do projeto. 
%Os objetivos específicos são apresentados a seguir:
%\begin{itemize}
%\item Explorar e comparar os algoritmos usados para conversão de séries temporais %em redes complexas;
%\item Estudar sistemas com base na análise de suas representações como redes %complexas.
%\end{itemize}

O objetivo geral é investigar a dinâmica de processos epidemiológicos em redes de mobilidade temporais e compará-la com a dinâmica em redes de mobilidade estáticas.

Os objetivos específicos são:

\begin{itemize}
    \item Realizar uma revisão de literatura sobre os métodos utilizados para resolver o problema
    abordado.
    \item Modelar processos epidemiológicos em redes de mobilidade.
    \item Simular os processos em redes temporais em diferentes escalas de tempo.
    \item Simular os processos em redes estáticas e comparar com o caso temporal.
    \item Contribuir com a divulgação de técnicas aplicadas à resolução do problema.
    \item Colaborar com a formação de recursos humanos especializados nesta área do
    conhecimento.
    \item Favorecer a consolidação do Laboratório de Computação de Sistemas Inteligentes
    (CSILab) da Universidade Federal de Ouro Preto.
\end{itemize}
