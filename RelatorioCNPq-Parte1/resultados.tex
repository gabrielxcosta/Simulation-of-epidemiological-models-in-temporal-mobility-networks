\chapter{Resultados e Discussões}

Em nossa pesquisa, investigamos a dinâmica de processos epidemiológicos em redes de mobilidade temporais e as comparamos com redes de mobilidade estáticas, variando as resoluções temporais para agregação.

Para realizar essas comparações, utilizamos métricas como a Norma de Frobenius \cite{Demmel1997} e a Correlação de Pearson \cite{Moretin2017} para avaliar as trajetórias do modelo SIR em cada subpopulação em um total de 12 resoluções temporais, variando desde a melhor resolução (baseline) até a pior (uma rede única para os 60 dias, calculada como a média dos fluxos de todas as matrizes de entrada e saída). Observamos uma diferença infíma nas normas das trajetórias, da ordem de $10^{-10}$ a $10^{-11}$, sugerindo consistência nas trajetórias epidemiológicas em diferentes resoluções temporais.

Durante o curso deste projeto, enfrentamos desafios significativos ao obter dados relevantes de mobilidade temporal de bases de dados nacionais e internacionais. A obtenção e preparação desses dados consumiram uma quantidade considerável de nosso tempo de pesquisa.

É importante ressaltar que este projeto é atualmente um trabalho em andamento. Até o momento, conseguimos atingir com sucesso os objetivos propostos, que incluem a modelagem de processos epidemiológicos em redes de mobilidade, simulação desses processos em redes temporais em diferentes escalas de tempo e comparação com o caso de redes estáticas. No entanto, nossa investigação está longe de estar concluída.

Na próxima fase da pesquisa, que será a continuação deste projeto de Iniciação Científica, planejamos comparar a modelagem de modelos epidemiológicos para dados de mobilidade com e sem a consideração de redes. Isso envolverá o agrupamento de populações fragmentadas em uma única população e a análise das correspondências e acurácia de usar um método em relação ao outro. Estamos ansiosos para explorar essas questões em nossa pesquisa contínua.