\chapter{Trabalhos relacionados}

Neste capítulo, faremos um levantamento de trabalhos que aplicaram modelos epidemiológicos em diversas redes, abrangendo diferentes contextos, como a propagação de doenças infecciosas e informações em sistemas complexos. Para cada trabalho revisado, destacaremos os modelos utilizados, as características das redes e os resultados obtidos. Nosso objetivo é oferecer uma visão ampla das contribuições recentes no campo da epidemiologia de redes e como essas pesquisas têm contribuído para entender e prever processos de propagação em sistemas complexos.

\section{Signal propagation in complex networks}

A pesquisa conduzida por \citeonline{JI20231} é uma contribuição notável e extremamente relevante, abrangendo diversos modelos e redes complexas. Explora modelos fundamentais, como os epidêmicos, de Kuramoto, passeios aleatórios, reação-difusão e percolação, que descrevem a dinâmica de hospedeiros ou patógenos em redes de contato e a propagação de sinais. Além disso, investiga fatores topológicos, como redes temporais e multicamadas, e apresenta um quadro teórico abrangente para entender a interação entre dinâmica e topologia, destacando estudos recentes sobre o controle de redes.

O trabalho também se aprofunda em técnicas contemporâneas para analisar a propagação de sinais a partir de dados observacionais, abrangendo transferência de informações, métodos de inteligência artificial e análise de séries temporais em redes. Isso engloba a reconstrução da estrutura da rede, a localização das fontes de sinais e a previsão dos links entre os nós da rede. Além disso, aborda a análise de séries temporais impulsionada por IA, usando técnicas de aprendizado de máquina para processar e prever essas séries temporais.

Por fim, a pesquisa destaca aplicações significativas em campos como epidemias, dinâmica social, neurociência, redes de transmissão elétrica e robótica. Essas aplicações evidenciam a relevância das técnicas e modelos apresentados para compreender a propagação de doenças, a disseminação de informações em redes sociais e a confiabilidade das redes de comunicação. Em síntese, o trabalho de \citeonline{JI20231} proporciona uma visão abrangente e valiosa da dinâmica de propagação de sinais em redes complexas.

\section{Small but slow world: how network topology and burstiness slow down spreading}

\citeonline{Karsai_2011} conduziram um estudo que se baseou em dados empíricos de sequências de contatos, utilizando o modelo SI para analisar a dinâmica temporal da propagação de informações em redes de comunicação, com dados provenientes do conjunto de dados \textit{Reality Mining} \cite{Eagle2006}. Eles aplicaram modelos nulos para discernir os fatores que afetam essa propagação, destacando as correlações entre peso e topologia, bem como os padrões de atividade dos indivíduos como principais influências na desaceleração da propagação. O estudo também comparou redes de chamadas móveis e registros de e-mails, identificando diferenças notáveis. Por exemplo, na rede de chamadas móveis, chamadas consecutivas para muitas pessoas em curtos períodos impulsionaram um aumento acentuado na prevalência, enquanto na rede de e-mails, hubs com alto grau de comunicação desempenharam um papel fundamental na rápida propagação. Além disso, a pesquisa explorou como os padrões diários de atividade afetam a dinâmica de propagação, considerando os pesos das conexões. Em resumo, o estudo proporciona valiosas perspectivas sobre como a topologia e a intermitência da rede podem influenciar a velocidade de propagação nas redes de comunicação.

\section{Networks and epidemic models}

Outro trabalho que foi de grande valia para o desenvolvimento de nossa pesquisa foi conduzido por \citeonline{keeling2005networks}. Foi realizado um estudo abrangente na área da epidemiologia de doenças infecciosas, explorando uma ampla gama de modelos e estruturas de redes. Os modelos investigados incluíram aproximações em pares, modelos compartimentais (como SIR e SEIR), modelos baseados em agentes e modelos estocásticos. Além disso, diferentes tipos de redes foram analisados, desde redes de mistura completa até redes aleatórias, sem escala e de mundo pequeno, incluindo também redes de contato e árvores de transmissão.

Vale ressaltar que essa lista não é exaustiva, visto que a escolha de modelos e redes depende dos objetivos específicos de cada estudo e dos dados disponíveis. Esta pesquisa enfatiza a aplicação de modelos matemáticos na simulação da propagação de doenças infecciosas em redes de contatos individuais, fornecendo insights cruciais para a previsão da dinâmica epidêmica, a avaliação de estratégias de intervenção e a identificação de fatores determinantes na transmissão de doenças.

Ademais, essa investigação destaca a importância de selecionar a estrutura de rede mais adequada às características do problema e aos dados disponíveis, seja ela uma rede de mistura completa, aleatória, sem escala ou outra. Os resultados obtidos por meio desses modelos baseados em redes têm contribuído de forma substancial para a compreensão dos padrões de transmissão de doenças e o desenvolvimento de estratégias eficazes de controle, incluindo a identificação de indivíduos denominados \textit{superpropagadores} e a avaliação do impacto de campanhas de vacinação direcionadas.

\section{Comparing metapopulation dynamics of infectious diseases under different models of human movement}

O trabalho do qual obtivemos modelos fundamentais para o progresso de nosso projeto é o de \citeonline{Citron2021}. Neste estudo, são discutidos modelos de metapopulações compartimentais que incorporam o movimento dos hospedeiros e a dinâmica das doenças infecciosas. Os autores empregaram três modelos de transmissão de doenças infecciosas: o modelo suscetível-infectado-recuperado (SIR), o modelo suscetível-infectado-suscetível (SIS) \cite{Kermack1927} e o modelo Ross-Macdonald \cite{SIMOY2020105452}.

O modelo SIS divide a população em dois compartimentos, suscetíveis e infectados, com a possibilidade de indivíduos transitar entre esses estados. O modelo Ross-Macdonald, mais complexo, inclui a transmissão de doenças infecciosas por vetores.

Embora esta investigação não se concentre especificamente na malária, ela menciona um estudo de caso relacionado à transmissão e importação de malária na Ilha de Bioko, na Guiné Equatorial \cite{Guerra2019}. Neste contexto, os autores destacam a importância de selecionar modelos de movimento adequados. Eles utilizaram dados de inquéritos de indicadores da malária (MIS) \cite{citron2020supporting} para mapear a prevalência estimada da malária e a probabilidade estimada de um residente deixar a ilha . A análise mostrou uma taxa de risco elevada de infecção entre pessoas que relataram viagens recentes.

O estudo aplicou o modelo Ross-Macdonald para estimar o $R_{0}$ local, usando dados de prevalência e comportamento de viagem. Dados populacionais e estimativas geoespaciais foram essenciais para determinar a prevalência em cada local. Além disso, foram utilizados os destinos de viagem que as pessoas relataram para aprimorar um modelo que analisa como escolhem seus destinos ao viajar. Em resumo, esta pesquisa destaca a importância de modelar com precisão a mobilidade humana no controle de doenças infecciosas, com ênfase na malária na Ilha de Bioko.